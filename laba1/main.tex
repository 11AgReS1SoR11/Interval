\documentclass[a4paper,12pt]{article}
\usepackage[utf8]{inputenc}
\usepackage[russian]{babel}
\usepackage{amsmath}
\usepackage{multicol}
\usepackage{url}
\usepackage[margin=1cm]{graphicx} % Required for inserting images
\usepackage[a4paper, top=2cm, bottom=3cm, left=2cm, right=2cm]{geometry}
\title{Санкт-Петербургский политехнический университет
Петра Великого
Физико-механический институт
Высшая школа прикладной математики и вычислительной
физики}
\date{}
\begin{document}

\maketitle
\begin{center}
{\fontsize{25}{}\selectfont Отчёт \\
по лабораторной работе №1 \\
по дисциплине \\
«Интервальный анализ»}

\end{center}
\vspace{3 cm}
\begin{flushright}
Выполнил студент:\\
Басалаев Даниил
группа:\\
5030102/10201\\

\end{flushright}

\vspace*{\fill} \begin{center}Санкт-Петербург\end{center}

\newpage % Начать новую страницу для содержания
\tableofcontents


\newpage
\section{Постановка задачи}

Пусть дана вещественная матрица:

\[
\text{midA} = \begin{bmatrix}
a_{11} & a_{12} \\
a_{21} & a_{22}
\end{bmatrix} \tag{1}
\]
и неотрицательное число

\[
\alpha \in [0, \max\{a_{ij}\}], \quad i, j = 1, 2 \tag{2}
\]
Рассмотрим матрицу радиусов:

\[
\text{radA} = \begin{bmatrix}
r_{11} & r_{12} \\
r_{21} & r_{22}
\end{bmatrix} \tag{3}
\]

Построим интервальную матрицу следующего вида:

\[
A = \begin{bmatrix}
[a_{11} - \alpha \cdot r_{(1,1)}, & a_{11} + \alpha \cdot r_{(1,1)}]
[a_{12} - \alpha \cdot r_{(1,2)}, & a_{12} + \alpha \cdot r_{(1,2)}] \\
[a_{21} - \alpha \cdot r_{(2,1)}, & a_{21} + \alpha \cdot r_{(2,1)}] 
[a_{22} - \alpha \cdot r_{(2,2)}, & a_{22} + \alpha \cdot r_{(2,2)}] \\
\end{bmatrix}, \quad i = 1, 2  \tag{4}
\]


Требуется найти \(\min\{\alpha \mid 0 \in \det A\}\). В целях конкретизации и возможности проверки решения будем использовать следующую матрицу:

\[
\text{midA} = \begin{bmatrix}
1.05 & 0.95 \\
1 & 1
\end{bmatrix} \tag{5}
\]

\begin{alignleft}
\section{Теоретическое обоснование}

\subsection{Понятие интервала}


Интервал:
\[
[a, b] := \{ x \in \mathbb{R} \mid a \leq x \leq b \} \tag{6}
\]
Середина интервала:
\[
\text{mid}[a, b] = \frac{1}{2} (a + b) \tag{7}
\]
Ширина интервала:
\[
\text{wid}[a, b] = b - a \tag{8}
\]
Радиус интервала:
\[
\text{rad}[a, b] = \frac{1}{2} (b - a) \tag{9}
\]



\subsection{Основные операции над интервалами}
Сумма интервалов:
\[
[a, b] + [c, d] = [a + c, b + d] \tag{10}
\]
Разность интервалов:
\[
[a, b] - [c, d] = [a - d, b - c] \tag{11}
\]
Умножение интервалов:
\[
[a, b] \cdot [c, d] = [\min(ac, ad, bc, bd), \max(ac, ad, bc, bd)] \tag{12}
\]
Деление интервалов:
\[
\frac{[a, b]}{[c, d]} = [\min(\frac{a}{c}, \frac{a}{d}, \frac{b}{c}, \frac{b}{d}), \max(\frac{a}{c}, \frac{a}{d}, \frac{b}{c}, \frac{b}{d})] \tag{13}
\]
Пусть \( \text{midA} = \{a_{ij}\}_{i,j\in N} \) – точечная вещественная матрица середин, \( \text{radA} = \{r_{ij}\}_{i,j\in N} \) – точечная вещественная матрица радиусов. Операцией midrad назовем следующую функцию:
\[
\text{midrad}(\text{midA}, \text{radA}) = \{[a_{ij} - r_{ij}, a_{ij} + r_{ij}]\}_{i,j\in N} \tag{14}
\]
Результатом операции является интервальная матрица.

\end{alignleft}


\subsection{Описание алгоритма}

    \item Инициализация алгоритма: положим интервал неопределённости \( \alpha \in [0, \alpha_{\text{max}}] \), где
    
\[
    \alpha_{\text{max}} = \max_{a_{i,j} \in \text{midA}} |a_{i,j}| \tag{15}
    \]
    Заметим, что при этом на левой границе при \( \alpha = 0 \) предполагается, что \( 0 \notin \det A(\alpha) \). В случае же, если это не так, алгоритм оптимизации сразу же выходит по очевидным причинам. Правая граница выбрана таковой в силу того, что выражение \( A(\alpha_{\text{max}}) \) будет содержать 0-ую точечную матрицу, которая по очевидным причинам удовлетворяет условию:
    
\[
0 \in \det A(\alpha_{\text{max}}) \tag{16}
\] 

Таким образом, при инициализации алгоритма условие выражения 19 на левой границе не выполнено, а на правой же оно выполнено. Это правило является основным при работе алгоритма. \\


Найдем допустимый интервал значений \(\alpha\). С помощью метода половинного деления будем сужать его до тех пор, пока не достигнем заданной точности \(\varepsilon = 10^{-10}\) — это условие конца алгоритма оптимизации. \\

Далее с помощью метода половинного деления производится сужение интервала значений \(\alpha\) до тех пор, пока не будет достигнута необходимая точность \(\varepsilon = 10^{-10}\). Если \(0\) входит в интервал определителя, то уменьшаем правую границу интервала. Если нет, то увеличиваем левую.


\begin{alignleft}

\section{Результаты}

Пусть дана вещественная матрица:
\[
\text{midA} = \begin{bmatrix}
1.05 & 0.95 \\
1 & 1
\end{bmatrix} \tag{17}
\]
и матрица радиусов:


\[
\text{radA} = \begin{bmatrix}
1 & 1 \\
1 & 1
\end{bmatrix} \tag{18}
\]
Текущие границы: \(\alpha \in [0, 1.05]\).\\

Начинаем итерационный процесс: \\
1. 

\[
A = \begin{bmatrix}
[0.525 & 1.575] 
[0.425 & 1.475]  \\
[0.475 & 1.525] 
[0.475 & 1.525] \\
\end{bmatrix}, \tag{19}
\]

Итоговый интервал из определителя \([-2.0, 2.2]\). Текущие границы: \(\alpha \in [0, 0.525]\). Целевое значение \(0\) попадает в итоговый интервал. Необходимая точность не достигнута, так что необходимо изменить правую границу интервала.

2.

\[
A = \begin{bmatrix}
[0.788 & 1.313]
[0.688 & 1.21] \\
[0.738 & 1.26]
[0.738 & 1.26]
\end{bmatrix} \tag{20}
\]

Итоговый интервал из определителя \([-0.949, 1.15]\). Текущие границы: \(\alpha \in [0, 0.263]\). Целевое значение \(0\) попадает в итоговый интервал. Необходимая точность не достигнута, так что необходимо изменить правую границу интервала.

3.

\[
A = \begin{bmatrix}
[0.919 & 1.181]
[0.819 & 1.081] \\
[0.869 & 1.131]
[0.869 & 1.131]
\end{bmatrix} \tag{21}
\]

Итоговый интервал из определителя \([-0.425, 0.625]\). Текущие границы: \(\alpha \in [0, 0.1313]\). Целевое значение \(0\) попадает в итоговый интервал. Необходимая точность не достигнута, так что необходимо изменить правую границу интервала.

4.

\[
A = \begin{bmatrix}
[1.025 & 1.075]
[0.925 & 0.975] \\
[0.975 & 1.025]
[0.975 & 1.025]
\end{bmatrix} \tag{22}
\]

Итоговый интервал из определителя \([-1.51 \times 10^{-10}, 0.200]\). Текущие границы: \(\alpha = 0.025 \pm 4 \times 10^{-10}\), что удовлетворяет заданной точности.


\end{alignleft}

\section{Обсуждение}

\subsection{Физическая интерпретация}

Матрица \(A\) представляет собой матрицу, сформированную при минимальном значении радиуса матричных элементов \(\alpha\). Это значение \(\alpha\) является критическим, поскольку при нем матрица \(A\) становится сингулярной, то есть теряет свою обратимость, и \(\det A = 0\). В такой точке система уравнений становится вырожденной, что означает наличие бесконечного множества решений. Отсутствие однозначного решения указывает на то, что часть данных либо не хватает, либо они содержат недостаточно информации для решения задачи.

\subsection{Компьютерная томография}

В компьютерной томографии основная задача состоит в восстановлении изображения на основе данных, полученных с помощью рентгеновских лучей, проходящих через объект. При решении обратных задач, подобных этой, возникает множество источников неопределенности: шумы в данных, ограниченная точность измерений и артефакты реконструкции. Для учета этих неопределенностей применяются интервальные методы. Вместо того чтобы считать параметры точно известными, их представляют в виде интервалов — как в постановке задачи с интервальной матрицей. \\


\section{Вывод}
Использование интервальной арифметики позволяет учитывать ошибки и неопределенности в данных и ограничить их влияние на результат реконструкции. Вместо точных величин рассматриваются интервалы, что делает алгоритмы более устойчивыми к шумам и ошибкам.


\section{GitHub}
Ссылка на репозиторий с кодом: \url{https://github.com/11AgReS1SoR11/Interval.git} \\


\section{Литература}
\begin{enumerate}
    \item A.Н.Баженов. Интервальный анализ. Основы теории и учебные примеры. СПБПУ. 2020.
    \item A.Н.Баженов., Н.В.Ермаков. Малоракурсная томография. Спб.: Политех-пресс. 2023.
\end{enumerate}

\end{document}
